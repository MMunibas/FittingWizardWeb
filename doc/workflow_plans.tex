\documentclass[12pt,a4paper]{article}
\usepackage[utf8]{inputenc}
\usepackage[english]{babel}
\usepackage[T1]{fontenc}

\usepackage{lmodern}

\usepackage{url}
\usepackage{hyperref}
\hypersetup{colorlinks=true,urlcolor=blue}

%\usepackage{listings}

\usepackage[top=3cm, bottom=3cm, left=2.5cm, right=2.5cm]{geometry}

\setlength\parindent{0pt}

\author{Florent Hédin, Prashant Gupta}
\title{FittingWizard : plans for further improvements}

\begin{document}

\maketitle

\section{LJ parameters fitting}

The goal would be to extend the current application in order to have the
possibility to use it for performing a fit of the Lennard-Jones parameters for
the CHARMM forcefield.

\medskip

The easiest way of doing that would be to add a button on the last panel of the 
current software, i.e. when the ESP fit is done. Then a new panel would appear 
for setting up parameters for the LJ fit.

\section{Roadmap}

Here is a first estimate of the required steps:

\begin{itemize}
\item Choose the type of properties used for estimating the LJ parameters: 
density, heat of vaporisation, ...

\item Generate CHARMM compatible coordinates (COR or PDB) and topology (PSF).

\item Atom types: may be give to the user the opportunity to edit the atom 
types manually before generating the COR and PSF files.

\item Initial guess for the well depth and distance parameters, and increment 
used: also define a lower and upper limit for the possible values ?

\item Generate the corresponding input files for CHARMM.

\item Copy files to distant server, similarly to what was done with Gaussian.

\item Run CHARMM on the distant server.

\item Either: (i) copy back output files on the local machine and run the 
scripts locally, or (ii) run those scripts on the distant server and just copy 
back final results to the local machine. Will depend on the computational cost 
required by those scripts.

\end{itemize}

\section{Technical considerations}

\subsection{Coordinates and topology files}
The current software uses XYZ coordinates, not read by CHARMM, and containing 
no topology or connectivity information.

\medskip

Then it is necessary to generate a COR or PDB file, and a topology PSF file. We 
currently have a script from V. Zoete for building a PSF, and need to check for 
some scripts written by group members for generating COR or PDB. 

\medskip

An alternative would be to have look at the possibility offered by 
\textbf{OpenBabel}.\\
(\href{http://openbabel.org/wiki/Babel}{http://openbabel.org/wiki/Babel}).

\subsection{CHARMM input files}
We need to have a proper template that will be used for preparing the input 
files. Ideally the user would be able to run the fitting without having to 
modify this template, but giving this opportunity to the user would be useful.


\end{document}

