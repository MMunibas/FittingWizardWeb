\documentclass[11pt,a4paper]{article}
\usepackage[utf8]{inputenc}
\usepackage[english]{babel}
\usepackage[T1]{fontenc}

\usepackage{lmodern}

\usepackage{graphicx}

\usepackage{url}
\usepackage{hyperref}
\hypersetup{colorlinks=true,urlcolor=blue}

\usepackage{listings}
\usepackage[top=3cm, bottom=3cm, left=2.5cm, right=2.5cm]{geometry}

\setlength\parindent{0pt}

\author{Florent Hédin}
\date{}
\title{FittingWizard v0.8.0 : software prerequisites and installation 
procedure}

\begin{document}

\maketitle

\section{Java version}
Oracle's Java proprietary implementation is required for running this software 
under Linux : this can be downloaded at :\\ 
\href{http://www.oracle.com/technetwork/java/javase/downloads/index.html}
{http://www.oracle.com/technetwork/java/javase/downloads/index.html}\\

Download the \texttt{JRE} if you just plan to use the software, or the 
\texttt{JDK} if you plan some Java software development.

\subsection{Fedora / RedHat Entreprise Linux / CentOS}
Check this page for instructions if you are using either Fedora 12 to 20 or 
RedHat/CentOS 5.10 to 6.5 : \\
\href{http://www.if-not-true-then-false.com/2010/install-sun-oracle-java-jdk-jre-7-on-fedora-centos-red-hat-rhel/}
{http://www.if-not-true-then-false.com/2010/install-sun-oracle-java-jdk-jre-7-on-fedora-centos-red-hat-rhel/}

Please follow all the steps of the procedure, and do not forget the step 
\texttt{alternatives --config java} for enabling the use of the Oracle 
implementation.\\

\subsection{Ubuntu}
Check this page for instructions if you are using Ubuntu :\\
\href{http://www.wikihow.com/Install-Oracle-Java-on-Ubuntu-Linux}
{http://www.wikihow.com/Install-Oracle-Java-on-Ubuntu-Linux}

\section{Babel/Open Babel}
Open Babel is a toolkit for manipulating chemical data : in this toolkit it is 
mainly used for chemical coordinates file conversion. Some Linux distributions 
provide binary compiled versions.\\

Otherwise check 
\href{http://openbabel.org/wiki/Main_Page}{http://openbabel.org/wiki/Main\_Page}
 for instructions and manual download and compiling.

\section{VMD compatibility}
This software allows you to visualize the resulting multipoles within VMD.

First you need to install VMD : 
\href{http://www.ks.uiuc.edu/Development/Download/download.cgi?PackageName=VMD}{Download
 here} and see 
\href{http://www.ks.uiuc.edu/Research/vmd/current/docs.html}{here} for 
installation instructions.\\

Then you need to compile a small fortran software provided in the 
\texttt{./scripts} subdirectory with the following command : 
\texttt{gfortran fieldcomp\_dyna.f90 -o fieldcomp}.

If \texttt{gfortran} is not installed, use \texttt{''yum install 
gcc-gfortran''} for Fedora/CentOS/RedHat or \texttt{''apt-get install 
gfortran''} for Ubuntu.

\section{Python 2.7}
You need to have a Python 2.7 scripting environment installed : this is 
already the case for most of the Linux distributions, if not please check the 
documentation pages of your distribution for more instructions.\\

Alternatively, you can download and compile Python 2.7 by yourself :\\
\href{http://www.python.org/download/releases/2.7/}{http://www.python.org/download/releases/2.7/}

\subsection{NumPy and SciPy}

NumPy and SciPy are two Python modules used for scientific and mathematics 
computations. Most of the Linux distributions will provide a way of directly 
installing binary versions of those libraries.\\

Otherwise, check \href{http://www.numpy.org/}{http://www.numpy.org/} and 
\href{http://www.scipy.org/}{http://www.scipy.org/} for more instructions or 
for a manual install.

\subsection{RDKit}
RDKit is an open source toolkit for cheminformatics. See 
\href{http://sourceforge.net/projects/rdkit/files/}{http://sourceforge.net/projects/rdkit/files/}
for downloading the sources and 
\href{http://rdkit.org/docs/index.html}{http://rdkit.org/docs/index.html}
for installation instructions.\\

\underline{A few remarks:}\\

If you need to compile RDKit by yourself, you will require several packages:
\begin{itemize}
\item Python development libraries : under Ubuntu, install 
\texttt{''apt-get install python-dev''}, 
under Fedora or CentOS/RHEL install \texttt{''yum install python-devel''}, if 
you compiled 
Python by yourself you should have it.

\item Flex and Bison and Cmake, 3 packages used for building the software suite:
\texttt{''yum install flex bison cmake''} or \texttt{''apt-get install flex 
bison cmake''}. If you need to install the doftware manually, check:
\begin{itemize}
\item 
\href{http://www.gnu.org/software/bison/}{http://www.gnu.org/software/bison/}
\item \href{http://flex.sourceforge.net/}{http://flex.sourceforge.net/}
\item 
\href{http://www.cmake.org/cmake/resources/software.html}{http://www.cmake.org/cmake/resources/software.html}
\end{itemize}

\item The boost library, a collection of tools commonly used in modern 
programming: install with \texttt{''yum install boost boost-devel''}
or \texttt{''apt-get install libboost-all-dev''}. Alternatively for a manual 
install, see 
\href{http://www.boost.org/users/download/}{http://www.boost.org/users/download/}
\end{itemize}


Before running the \texttt{''make''} command and after using 
\texttt{''cmake''} (see 
RDKit docs), run \texttt{''cmake -D RDK\_INSTALL\_INTREE=OFF .''} for 
choosing as installation path the default \texttt{/usr/local/}\\

Once the software is installed you need to define the two following 
environment variables :
\texttt{export PYTHONPATH=/usr/local/lib64/python2.7/site-packages/} \\
\texttt{export LD\_LIBRARY\_PATH=/usr/local/lib/} \\

Then open a \texttt{python2.7} intepreter and try the following two commands 
to test the installation: \\
\texttt{import rdkit}\\
\texttt{from rdkit import Chem}\\

If no error message is displayed \texttt{RDKit} can be properly used by 
\texttt{FittingWizard}



\end{document}

