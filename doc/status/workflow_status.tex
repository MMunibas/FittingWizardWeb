\documentclass[12pt,a4paper]{article}
\usepackage[utf8]{inputenc}
\usepackage[english]{babel}
\usepackage[T1]{fontenc}

\usepackage{lmodern}

\usepackage{url}
\usepackage{hyperref}
\hypersetup{colorlinks=true,urlcolor=blue}

\usepackage{graphicx}

\usepackage{ulem} % for \sout command

\date{}

\usepackage[top=3cm, bottom=3cm, left=2.5cm, right=2.5cm]{geometry}

\setlength\parindent{0pt}

\author{Florent H\'{e}din}
\title{FittingWizard : Status of the current implementation}

\begin{document}

\maketitle

\tableofcontents

\section{Overview of the workflow for the current version}

\subsection{Welcome screen}

At initial run, welcome screen for choosing what to do : Figure \ref{fig0} \\

%\medskip

\begin{figure}[h!]
\centering
\includegraphics[width=0.9\linewidth]{pics/scr0}
\caption{Welcome screen}
\label{fig0}
\end{figure}

\subsection{MTP parameters from Ab Initio}

\subsubsection{Loading coordinates}

First selection in welcome screen gives access to the Gaussian optimization procedure, where first 
a molecule 
is loaded : Figure \ref{fig1} \\

%\medskip

\begin{figure}[h!]
\centering
\includegraphics[width=0.9\linewidth]{pics/scr1}
\caption{Loading molecule}
\label{fig1}
\end{figure}

\subsubsection{MTP fit}

Then after success of the Ab Initio calculation the user defines initial charges for the atoms, 
chooses fit parameters (Figure \ref{fig2}) and runs the optimization. \\

\begin{figure}[h!]
\centering
\includegraphics[width=0.9\linewidth]{pics/scr2}
\caption{Choosing fit parameters}
\label{fig2}
\end{figure}

After a short time a table displays the results : Figure \ref{fig3} \\

\begin{figure}[h!]
\centering
\includegraphics[width=0.9\linewidth]{pics/scr3}
\caption{Visualizing results in table}
\label{fig3}
\end{figure}

It is then possible to go to the CHARMM fit section by using the bottom right button.

\clearpage

\subsection{CHARMM fitting procedure}

From the Welcome screen (Figure \ref{fig0}) user can access directly to the CHARMM fitting 
procedure (useful if the previous section was already performed before). Otherwise the screen is 
the next step reached in the workflow : Figure \ref{fig4} \\

\begin{figure}[h!]
\centering
\includegraphics[width=0.9\linewidth]{pics/scr4}
\caption{CHARMM fitting procedure : choose files}
\label{fig4}
\end{figure}

\subsubsection{Generating input files}

Then input files can be generated, and they are displayed in several tabs, and if required the user 
can review and edit some of those files. When ready the user can submit all the simulations by 
pressing the bottom right button that will internally call all the required scripts : Figure 
\ref{fig5} \\

\begin{figure}[h!]
\centering
\includegraphics[width=0.9\linewidth]{pics/scr5}
\caption{CHARMM fitting procedure : all input files generated and editable if required}
\label{fig5}
\end{figure}

\subsubsection{Visualizing output files}

If for some reason simulation failed an error window appears where the user can visualize failing 
output files : Figure \ref{fig6} \\

\begin{figure}[h!]
\centering
\includegraphics[width=0.9\linewidth]{pics/scr6}
\caption{CHARMM fitting procedure : error window}
\label{fig6}
\end{figure}

Otherwise at the end of all simulations the user gets a successful run window : Figure \ref{fig7}\\

\begin{figure}[h!]
\centering
\includegraphics[width=0.9\linewidth]{pics/scr7}
\caption{CHARMM fitting procedure : success window}
\label{fig7}
\end{figure}

\subsubsection{Obtaining thermodynamic properties}

Then output files are parsed for extracting for extracting Temperature, molar mass and number of 
residues and then the density, Enthalpy of Vaporization and Free Energy of Solvation are estimated 
: Figure \ref{fig8}\\

\begin{figure}[h!]
\centering
\includegraphics[width=0.9\linewidth]{pics/scr8}
\caption{CHARMM fitting procedure : obtaining the calculated thermodynamic properties}
\label{fig8}
\end{figure}

\subsubsection{Analyze CHARMM output files}

All the output files from a given simulation are stored in the \texttt{data} directory (it possible 
to 
change the path of this directory and its name from the \texttt{\detokenize{config_gui.ini}} file).
By choosing the option ``Analyze CHARMM output files'' from the main window (Figure \ref{fig0}) it 
is 
possible to reload existing output files from a subdirectory of \texttt{data} (see Figure 
\ref{fig8b}). Once loaded the 
files will bring the user again to the window of Figure \ref{fig8} for obtaining again all the 
estimated properties.

\begin{figure}[h!]
\centering
\includegraphics[width=0.9\linewidth]{pics/scr17}
\caption{CHARMM fitting procedure : \textit{a posteriori} extraction of data from existing output 
files from a previous run}
\label{fig8b}
\end{figure}

\clearpage

\subsection{Running a set of CHARMM simulations with altered PAR and TOP files}

\subsubsection{Define list of scaling parameters}

The third possibility in Welcome screen (Figure \ref{fig0}) gives the user the possibility to 
generate a set of scaled PAR and TOP files for CHARMM where the LJ parameters $\sigma$ and 
$\epsilon$ are scaled : the user first chooses how many modifications to perform : Figure 
\ref{fig9}\\

\begin{figure}[h!]
\centering
\includegraphics[width=0.9\linewidth]{pics/scr9}
\caption{CHARMM fitting procedure : defining number of grid values}
\label{fig9}
\end{figure}

Then the user chooses the scaling parameters : Figure \ref{fig10}\\

\begin{figure}[h!]
\centering
\includegraphics[width=0.9\linewidth]{pics/scr10}
\caption{CHARMM fitting procedure : defining grid values}
\label{fig10}
\end{figure}

\subsubsection{Save all files and run all simulations}

Then it is possible to save all those files an to run all simulations : Figure \ref{fig11}

\begin{figure}[h!]
\centering
\includegraphics[width=0.9\linewidth]{pics/scr11}
\caption{CHARMM fitting procedure : saving FF files and running}
\label{fig11}
\end{figure}

\clearpage

\subsection{Generating custom PSF and TOP files using a XYZ file}

The fourth possibility in Welcome screen (Figure \ref{fig0}) gives the user the possibility to 
generate a set of custom topology (TOP,RTF) and structure (PSF) files.\\

Those files are specifically generated for the studied molecule, using the optimised XYZ 
coordinates from the Gaussian simulation.\\

Once generated the files are displayed in a tabs structure (Figures \ref{fig15} and \ref{fig16})
so the user still has the possibility to complete the generated files.

\begin{figure}[h!]
\centering
\includegraphics[width=0.9\linewidth]{pics/scr15}
\caption{CHARMM fitting procedure : generating a TOP/RTF file}
\label{fig15}
\end{figure}

\begin{figure}[h!]
\centering
\includegraphics[width=0.9\linewidth]{pics/scr16}
\caption{CHARMM fitting procedure : generating a PSF file}
\label{fig16}
\end{figure}


\clearpage

\subsection{Database of chemical compounds}

The fifth possibility in Welcome screen (Figure \ref{fig0}) gives access to a database of 
compounds where the user can obtain experimental values for the density, Enthalpy of vaporization 
and Free Energy of solvation.\\

Compounds can be searched by : 

\begin{itemize}
\item By name : Figure \ref{fig12}
\item By formula : Figure \ref{fig13}
\item By SMILES notation : Figure \ref{fig14}
\end{itemize}

\begin{figure}[h!]
\centering
\includegraphics[width=0.9\linewidth]{pics/scr12}
\caption{Database of compounds : search by name}
\label{fig12}
\end{figure}

\begin{figure}[h!]
\centering
\includegraphics[width=0.9\linewidth]{pics/scr13}
\caption{Database of compounds : search by formula}
\label{fig13}
\end{figure}

\begin{figure}[h!]
\centering
\includegraphics[width=0.9\linewidth]{pics/scr14}
\caption{Database of compounds : search by SMILES notation}
\label{fig14}
\end{figure}

\clearpage

%-------------------------------------------------
%-------------------------------------------------

%\section{Future improvements}
%
%...

%-------------------------------------------------
%-------------------------------------------------

\section{Database details}

We provide as an experimental feature the access to a database of 
mass -- density -- Enthalpy of Vaporization -- Free Energy of solvation, where the user can search 
for a molecule and get those values for then comparing to results of the CHARMM simulations.\\

The user can search molecules by name, formula or SMILES notation 
\href{http://en.wikipedia.org/wiki/Simplified_molecular-input_line-entry_system}{Wikipedia 
details}.\\

Mass, Density come from \href{https://pubchem.ncbi.nlm.nih.gov/search/}{PubChem}, Free Energy by 
results provided by David Mobley et al. (\href{http://escholarship.org/uc/item/6sd403pz}{Link})\\

The current database was designed using the \href{http://en.wikipedia.org/wiki/SQL}{SQL} 
language, see Figure \ref{dbFig} for a description of the tables an relationship between them.\\

The database is accessed either by : 
\begin{itemize}
\item As an embedded file provided with the software that is accessed using the 
\href{http://www.sqlite.org/}{SQLite} software.
\item Or as a \href{http://www.mysql.com/}{MySQL} database that is hosted on a server and then 
accessed by Local or Internet connection.
\end{itemize}

The server database is read only, and the local database should be read/write (for the moment only 
read access is provided), so that the user can easily add new molecules, add missing values, 
etc...\\

Ideas of improvements :
The user should have the possibility to ``propose'' new compounds or new values to the 
development team, then the online server-hosted database could grow interactively (of course after 
a checking procedure).

\begin{figure}[h!]
\centering
\includegraphics[width=0.9\linewidth]{pics/db}
\caption{Database of compounds : Design and relations}
\label{dbFig}
\end{figure}

%-------------------------------------------------
%-------------------------------------------------

%\section{List of known Bugs}
%
%\subsection{From the GUI}
%
%\begin{itemize}
%\item \sout{ 
%PDB : The Gaussian coordinates are saved to a XYZ file, but for the CHARMM simulation 
%we need 
%a PDB or COR file : unfortunately PDB files generated from VMD or OpenBabel are not readable by 
%CHARMM directly. A code for converting the XYZ file to PDB directly from the code is currently 
%being written, it will reused code from the section generating RTF and PSF files (from Figures 
%\ref{fig15} and \ref{fig16}). 
%}
%
%\item \textbf{This is now solved} : code was included inside the fittingWizard for generating 
%internally PDB 
%/ COR / PSF CHARMM files directly from the software. The CHARMM atom typing for the atoms is also 
%done using an internal code, although it may fail assigning a type for some of them, but the user 
%has the possibility to manually edit those types.
%
%
%\end{itemize}
%
%\subsection{From Tristan's python scripts}
%
%Found by Florent :
%\begin{itemize}
%\item PRNLEV 2 in input file while using MPI is problematic and unnecessary as this is already the 
%default with newer versions of CHARMM, because the I/O is messy and then difficult to parse later.
%\item with CHARMM c40a2 it is necessary to increase the value of ECHECK which is bt default 20.0 
%(maximum energy fluctuation between 2 steps allowed by CHARMM, it can go over 20. with some scaled 
%epsilon/sigma : maybe minimisation should be longer ?)
%\item ssh submission fails if the private keys are encrypted : need to use ones which are not 
%password protected. but this is unsafe, need to check python reference of this ssh module for an 
%alternative. 
%\end{itemize}
%
%Found by Krystel : 
%\begin{itemize}
%\item the use of IC in the topology file (top,rtf) may lead to incorrect results.
%\end{itemize}

%-------------------------------------------------
%-------------------------------------------------

%\section{Outlook}
%
%\begin{itemize}
%\item For the moment the interface is blocked (waiting screen) as long as jobs are
%running : not convenient
%
%\item Linked to previous point : add possibility to resume properly a previous
%session ("data serialization" / possibility to save the whole working session)
%
%\item Add a proper comparison (with graphs ?) of exp. vs. calculated values
%
%\item Add the last section that will modify the epsilon parameters with a fit and run again 
%simulations.
%\end{itemize}

%-------------------------------------------------
%-------------------------------------------------


\section{Results}

%\subsection{Phenol}

%Experimental results for phenol :
%
%\begin{itemize}
%\item $\rho$ (g/cm$^-3$) : 1.073
%\item $\Delta H$ (kcal/mol) : 13.75
%\item $\Delta G$ (kcal/mol) : -6.60
%\end{itemize}
%
%\medskip
%
%Simulation: $\epsilon$ and $\sigma$ were scaled by multiplying by $1.00$ or $1.05$ or $1.10$
%
%\begin{center}
%\begin{tabular}{|c|c|c|c|c|}
%\hline  &  & $\sigma$*1.00 & $\sigma$*1.05 & $\sigma$*1.10 \\ 
%\hline $\epsilon*1.00$ & $\rho$ (g/cm$^-3$) & 1.099 & 0.9934 & 0.9004 \\ 
%\hline  & $\Delta H$ (kcal/mol) & 19.871 & 18.4993 & 17.563 \\ 
%\hline  & $\Delta G$ (kcal/mol) & -6.496 & -12.78375 & -28.96785 \\ 
%\hline  &  &  &  &  \\ 
%\hline $\epsilon*1.05$ & $\rho$ (g/cm$^-3$) & 1.093 & 0.9973 & 0.9068 \\ 
%\hline  & $\Delta H$ (kcal/mol) & 20.19 & 18.8933 & 18.0134 \\ 
%\hline  & $\Delta G$ (kcal/mol) & -3.76869 & -13.69526 & -30.26755 \\ 
%\hline  &  &  &  &  \\ 
%\hline $\epsilon*1.10$ & $\rho$ (g/cm$^-3$)& 1.099 & 0.9951 & 0.8999 \\ 
%\hline  & $\Delta H$ (kcal/mol) & 20.604 & 18.3849 & 17.67194 \\ 
%\hline  & $\Delta G$ (kcal/mol) & -3.74234 & -14.03508 & -32.66905 \\
%\hline  &  &  &  &  \\ 
%\hline 
%\end{tabular} 
%\end{center}
%
%Discussion : \\
%
%
%The evolution of $\rho$ and $\Delta H$ looks coherent, but the estimates of 
%$\Delta G$ are, for scaled  $\sigma$, really reaching non realistic values. One need to 
%investigate 
%scaling factors lower than one to see if such things also happen in that case.\\
%
%
%
%Nevertheless, all the $\rho$ - $\Delta H$, and also the $\Delta G$ for $\sigma$*1.00, are similar 
%to results obtained by from Prashant (unpublished work : Thermodynamic study of the intercalation 
%process for various
%pharmacaphore in the chromatographic system).

\clearpage

%-------------------------------------------------

\subsection{NMA}
Figures \ref{nma_val} \ref{nma_dens} \ref{nma_dh} \ref{nma_dg} show results obtained with current 
version of the fittingWizard when working with the NMA molecule.\\

With Figure \ref{nma_val} one can see a summary of the 3 computed properties : 

\begin{itemize}
\item Density $\rho$
\item Enthalpy of vaporization $\Delta H$
\item Free energy of solvation $\Delta G$
\end{itemize}

As experimental reference values from PubChem and the Mobley DB were used (see ``Database details'' 
section for links and references).\\

We also compare to results from Cazade \& Bereau \& Meuwly published in 
\href{http://pubs.acs.org/doi/abs/10.1021/jp5011692}{JPCB} \\

One can see in the table at the top of Figure \ref{nma_val} that $\rho$ and $\Delta H$ are always 
really close to what is found by experiments, and 
that $\Delta G$ is underestimated by 3 kcal/mol. Results are almost the same than the one published 
in JPCB, as same paramters and similar procedure/scripts were used.\\

By having a look at the 2 lower tables of \ref{nma_val} one can see results obtained when scaling 
all the $\sigma$ and $\epsilon$ parameters by $1.025~1.05~1.075~1.1$, again results are really 
similar to what was published, small differences and the lack of error estimate come from the fact 
that only one simulation only was running for each set of scaled parameters, as this was just a 
validation.\\

Figures \ref{nma_dens} to \ref{nma_dg} show graphical evolution of the estimated properties 
according to the Scaling factor, and are just a way of visualizing values from the previous tables.

\begin{figure}[h!]
\centering
\includegraphics[width=0.9\linewidth]{pics/nma}
\caption{Computed values with current version of the Wizard compared to previously published ones, 
for NMA}
\label{nma_val}
\end{figure}

\begin{figure}[h!]
\centering
\includegraphics[width=0.9\linewidth]{pics/nma_dens}
\caption{Density $\rho$ evolution for NMA, for different scaling parameters applied to the LJ 
$\sigma$ and $\epsilon$}
\label{nma_dens}
\end{figure}

\begin{figure}[h!]
\centering
\includegraphics[width=0.9\linewidth]{pics/nma_dh}
\caption{Enthalpy of vaporization $\Delta H$ evolution for NMA, for different scaling parameters 
applied to the LJ $\sigma$ and $\epsilon$}
\label{nma_dh}
\end{figure}

\begin{figure}[h!]
\centering
\includegraphics[width=0.9\linewidth]{pics/nma_dg}
\caption{Free energy of solvation $\Delta G$ evolution for NMA, for different scaling parameters 
applied to the LJ $\sigma$ and $\epsilon$}
\label{nma_dg}
\end{figure}

\clearpage

%-------------------------------------------------

\subsection{Benzonitrile}

%-------------------------------------------------

A second validation was performed for Benzonitrile : experimental parameters from Pubchem and 
the Mobley DB are in Table \ref{tab:benzonitrile_ref}:\\

\begin{table}[h!]
\centering
\begin{tabular}{|c|c|}
\hline  & Values (g/cm$^3$ and kcal/mol) \\ 
\hline $\rho$ & 1.0093 \\ 
\hline $\Delta H$ & 10.639 \\ 
\hline $\Delta G$ & -4.10 \\ 
\hline 
\end{tabular} 
\caption{Experimental values for Benzonitrile}
\label{tab:benzonitrile_ref}
\end{table}

Using the fittingWizard we obtained two set of values, see Table \ref{tab:benzonitrile_new} : \\

\begin{table}[h!]
\centering
\begin{tabular}{|c|c|c|}
\hline  & Run 1 & Run 2 \\ 
\hline $\rho$ & 0.998 & 1.0035 \\ 
\hline $\Delta H$ & 19.28 & 16.02 \\ 
\hline $\Delta G$ & -3.90 & -4.74 \\ 
\hline 
\end{tabular} 
\caption{Experimental values for Benzonitrile}
\label{tab:benzonitrile_new}
\end{table}

The difference between the Run 1 and Run2 was in the choice of the MTP parameters obtained from the 
fit from Gaussian : because the $\Delta H$ was twice what is expected for Run 1, a more careful 
choice was made for Run 2, by using different starting values for the fit, and by trying to 
minimize further the RMSE for each multipole term. In the end it reduced $\Delta H$ from 19 to 16 
which is still too high but closer to ref. from Table \ref{tab:benzonitrile_ref}.\\

\subsection{Conclusion : On the choice of the MTP and LJ parameters and their influence on the 
estimated properties}

One should notice that density is really not sensitive to a change of the MTP initial parameters 
(less than one percent of fluctuation in Table \ref{tab:benzonitrile_ref}), but that $\Delta G$ and 
$\Delta H$ are more impacted (18 \% and 17\% of fluctuations respectively).\\

When scaling the LJ parameters as done fo the NMA example, we notice that the $\Delta H$ shows 
the smallest relative fluctuations (maximum 4\%). But for $\rho$ and $\Delta G$ the observed 
fluctuations are for the worst case of 15\% and 20\% respectively.\\

Thus it seems that an approach combining both LJ scaling and a careful choice of the MTP parameters 
is necessary in order to fit as closely as possible to the experimental results.

\end{document}
