\documentclass[12pt,a4paper]{article}
\usepackage[utf8]{inputenc}
\usepackage[english]{babel}
\usepackage[T1]{fontenc}

\usepackage{lmodern}

%\usepackage{url}
%\usepackage{hyperref}
%\hypersetup{colorlinks=true,urlcolor=blue}

\usepackage{graphicx}

\usepackage[top=3cm, bottom=3cm, left=2.5cm, right=2.5cm]{geometry}

\setlength\parindent{0pt}

\author{Florent Hédin}
\title{FittingWizard : Status of the current implementation}

\begin{document}

\maketitle

\tableofcontents

\section{Overview of the workflow for the current version}

\subsection{Welcome screen}

At initial run, welcome screen for choosing what to do : Figure \ref{fig0} \\

%\medskip

\begin{figure}[h!]
\centering
\includegraphics[width=0.9\linewidth]{pics/scr0}
\caption{Welcome screen}
\label{fig0}
\end{figure}

\subsection{MTP parameters from Ab Initio}

\subsubsection{Loading coordinates}

First selection in welcome screen gives access to the Gaussian optimization procedure, where first 
a molecule 
is loaded : Figure \ref{fig1} \\

%\medskip

\begin{figure}[h!]
\centering
\includegraphics[width=0.9\linewidth]{pics/scr1}
\caption{Loading molecule}
\label{fig1}
\end{figure}

\subsubsection{MTP fit}

Then after success of the Ab Initio calculation the user defines initial charges for the atoms, 
chooses fit parameters (Figure \ref{fig2}) and runs the optimization. \\

\begin{figure}[h!]
\centering
\includegraphics[width=0.9\linewidth]{pics/scr2}
\caption{Choosing fit parameters}
\label{fig2}
\end{figure}

After a short time a table displays the results : Figure \ref{fig3} \\

\begin{figure}[h!]
\centering
\includegraphics[width=0.9\linewidth]{pics/scr3}
\caption{Visualizing results in table}
\label{fig3}
\end{figure}

It is then possible to go to the CHARMM fit section by using the bottom right button.

\clearpage

\subsection{CHARMM fitting procedure}

From the Welcome screen (Figure \ref{fig0}) user can access directly to the CHARMM fitting 
procedure (useful if the previous section was already performed before). Otherwise the screen is 
the next step reached in the workflow : Figure \ref{fig4} \\

\begin{figure}[h!]
\centering
\includegraphics[width=0.9\linewidth]{pics/scr4}
\caption{CHARMM fitting procedure : choose files}
\label{fig4}
\end{figure}

\subsubsection{Generating input files}

Then input files can be generated, and they are displayed in several tabs, and if required the user 
can review and edit some of those files. When ready the user can submit all the simulations by 
pressing the bottom right button that will internally call all the required scripts : Figure 
\ref{fig5} \\

\begin{figure}[h!]
\centering
\includegraphics[width=0.9\linewidth]{pics/scr5}
\caption{CHARMM fitting procedure : all input files generated and editable if required}
\label{fig5}
\end{figure}

\subsubsection{Visualizing output files}

If for some reason simulation failed an error window appears where the user can visualize failing 
output files : Figure \ref{fig6} \\

\begin{figure}[h!]
\centering
\includegraphics[width=0.9\linewidth]{pics/scr6}
\caption{CHARMM fitting procedure : error window}
\label{fig6}
\end{figure}

Otherwise at the end of all simulations the user gets a successful run window : Figure \ref{fig7}\\

\begin{figure}[h!]
\centering
\includegraphics[width=0.9\linewidth]{pics/scr7}
\caption{CHARMM fitting procedure : success window}
\label{fig7}
\end{figure}

\subsubsection{Obtaining thermodynamic properties}

Then output files are parsed for extracting for extracting Temperature, molar mass and number of 
residues and then the density, Enthalpy of Vaporization and Free Energy of Solvation are estimated 
: Figure \ref{fig8}\\

\begin{figure}[h!]
\centering
\includegraphics[width=0.9\linewidth]{pics/scr8}
\caption{CHARMM fitting procedure : obtaining the calculated thermodynamic properties}
\label{fig8}
\end{figure}

\clearpage

\subsection{Running a set of CHARMM simulations with altered PAR and TOP files}

\subsubsection{Define list of scaling parameters}

The third possibility in Welcome screen (Figure \ref{fig0}) gives the user the possibility to 
generate a set of scaled PAR and TOP files for CHARMM where the LJ parameters $\sigma$ and 
$\epsilon$ are scaled : the user first chooses how many modifications to perform : Figure 
\ref{fig9}\\

\begin{figure}[h!]
\centering
\includegraphics[width=0.9\linewidth]{pics/scr9}
\caption{CHARMM fitting procedure : defining number of grid values}
\label{fig9}
\end{figure}

Then the user chooses the scaling parameters : Figure \ref{fig10}\\

\begin{figure}[h!]
\centering
\includegraphics[width=0.9\linewidth]{pics/scr10}
\caption{CHARMM fitting procedure : defining grid values}
\label{fig10}
\end{figure}

\subsubsection{Save all files and run all simulations}

Then it is possible to save all those files an to run all simulations : Figure \ref{fig11}

\begin{figure}[h!]
\centering
\includegraphics[width=0.9\linewidth]{pics/scr11}
\caption{CHARMM fitting procedure : saving FF files and running}
\label{fig11}
\end{figure}

\clearpage

\subsection{Database of chemical compounds}

The fourth possibility in Welcome screen (Figure \ref{fig0}) gives access to a database of 
compounds where the user can obtain experimental values for the density, Enthalpy of vaporization 
and Free Energy of solvation.\\

Compounds can be searched by : 

\begin{itemize}
\item By name : Figure \ref{fig12}
\item By formula : Figure \ref{fig13}
\item By SMILES notation : Figure \ref{fig14}
\end{itemize}

\begin{figure}[h!]
\centering
\includegraphics[width=0.9\linewidth]{pics/scr12}
\caption{Database of compounds : search by name}
\label{fig12}
\end{figure}

\begin{figure}[h!]
\centering
\includegraphics[width=0.9\linewidth]{pics/scr13}
\caption{Database of compounds : search by formula}
\label{fig13}
\end{figure}

\begin{figure}[h!]
\centering
\includegraphics[width=0.9\linewidth]{pics/scr14}
\caption{Database of compounds : search by SMILES notation}
\label{fig14}
\end{figure}

\clearpage

%-------------------------------------------------
%-------------------------------------------------

\section{Future improvements}

...

%-------------------------------------------------
%-------------------------------------------------

\section{Database details}

\begin{figure}[h!]
\centering
\includegraphics[width=0.9\linewidth]{pics/db}
\caption{Database of compounds : Design and relations}
\label{dbFig}
\end{figure}

%-------------------------------------------------
%-------------------------------------------------

\section{List of known Bugs}

\subsection{From the GUI}

...

\subsection{From Tristan's python scripts}

Found by Florent :
\begin{itemize}
\item PRNLEV 2 in input file while using MPI is problematic and unnecessary as this is already the 
default with newer versions of CHARMM, because the I/O is messy and then difficult to parse later.
\item with CHARMM c40a2 it is necessary to increase the value of ECHECK which is bt default 20.0 
(maximum energy fluctuation between 2 steps allowed by CHARMM, it can go over 20. with some scaled 
epsilon/sigma : maybe minimisation should be longer ?)
\item ssh submission fails if the private keys are encrypted : need to use ones which are not 
password protected. but this is unsafe, need to check python reference of this ssh module for an 
alternative. 
\end{itemize}

Found by Krystel : 
\begin{itemize}
\item the use of IC in the topology file (top,rtf) may lead to incorrect results.
\end{itemize}

%-------------------------------------------------
%-------------------------------------------------

\section{Results}

\subsection{Phenol}



\end{document}
























